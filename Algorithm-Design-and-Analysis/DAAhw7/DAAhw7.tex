\documentclass[12pt,letterpaper]{article}
\usepackage{fullpage}
\usepackage[top=2cm, bottom=4.5cm, left=2.5cm, right=2.5cm]{geometry}
\usepackage{amsmath,amsthm,amsfonts,amssymb,amscd}
\usepackage{lastpage}
\usepackage{enumerate}
\usepackage{fancyhdr}
\usepackage{mathrsfs}
\usepackage{xcolor}
\usepackage{graphicx}
\usepackage{listings}
\usepackage{hyperref}
\usepackage{mathtools}
\usepackage{xfrac}
\usepackage{algorithm}
\usepackage[noend]{algpseudocode}
\usepackage{bbm}

\hypersetup{
  colorlinks=true,
  linkcolor=blue,
  linkbordercolor={0 0 1}
}
\linespread{1.1}
 
\renewcommand\lstlistingname{Algorithm}
\renewcommand\lstlistlistingname{Algorithms}
\def\lstlistingautorefname{Alg.}


\lstdefinestyle{Python}{
    language        = Python,
    frame           = lines, 
    basicstyle      = \footnotesize,
    keywordstyle    = \color{blue},
    stringstyle     = \color{green},
    commentstyle    = \color{red}\ttfamily
}

\setlength{\parindent}{0.0in}
\setlength{\parskip}{0.05in}

\newcommand\course{Algorithms}
\newcommand\NetIDa{SUN Yilin}
\newcommand\NetIDb{520030910361}

\pagestyle{fancyplain}
\headheight 35pt
\lhead{\NetIDa}
\lhead{\NetIDa\\\NetIDb}
\chead{\textbf{\Large Edmonds' Blossom Algorithm }}
\rhead{\course \\ \today}
\lfoot{}
\cfoot{}
\rfoot{\small\thepage}
\headsep 1.5em

\begin{document}

\section{}
\subsection{}
Proof of $M$ is a maximum matching $\implies$ no $M$-augmenting path exists:\\
If there is an $M$-augmenting path denoted by $P=e_1e_2e_3\dots e_n$,
where $e_i=(v_{i-1},v_i), \forall i\in [n]$.
Then by definition of augmenting path we know that $e_1\notin M, e_n\notin M$.
It follows that, by definition of $M$-alternating path, 
$E_{even}=\{e_2,e_4,\dots,e_{n-1}\}\in M$
and $E_{odd}=\{e_1,e_3,e_5,\dots,e_{n-2},e_{n}\}\notin M$.
Then simply by switching the edges between $E_{even}$ and $E_{odd}$
and construct another matching $M'=(M\backslash E_{even})\cup E_{odd}$,
we would be able to construct a larger matching $M'$,
because obviously $|E_{even}|<|E_{odd}|$.\\
\subsection{}
Proof of $M$ is a maximum matching $\impliedby$ no $M$-augmenting path exists:\\
If $M$ is not a maximum matching, following the hint,
let $M'$ be a maximum matching such that $|M\cap M'|$ is maximized,
we now consider the subgraph $H=M\cup M'$ of $G$.
Here when constructing the subgraph we do a subtle but important modification that 
\textbf{for any edge that is both in $M$ and $M'$ we count it twice.}
We want to analyse the connected components of $H$.
The degree of any vertex in $H$ is at most 2 
because any vertex has at most two edges incident to it.
Thus the connected components of $H$ is either a single vertex,
a path, or a cycle.
We now consider the edge distribution of these cases respectively.\\
\textbf{Case 1: Single vertex.}\\
Trivial since no edges in these components.\\
\textbf{Case 2: Cycle.}\\
Any cycle of $H$ must be even length otherwise there will be 
two edges adjacent to each other and from the same matching.
It follows that any cycle must contain exactly the same number of edges from $M$ and $M'$.\\
\textbf{Case 3: Path.}\\
The paths can be either even length or odd length.
For those even length path, like the cycles,
they also contain same number of edges from $M$ and $M'$.
For those odd length path denoted by $P=e_1e_2e_3\dots e_n$,
we define $E_{even}=\{e_2,e_4,\dots,e_{n-1}\}$
and $E_{odd}=\{e_1,e_3,e_5,\dots,e_{n-2},e_{n}\}$. 
Then either $E_{even}\in M, E_{odd}\in M'$ or $E_{even}\in M', E_{odd}\in M$.\\
\newline
From the above three cases, the only case where a connected component
contains more edges from $M'$ than $M$ is in case 3 where 
there is a path $P$ with odd length and $E_{odd}\in M', E_{even}\in M$.
Since we know $M'$ is a maximum matching, $|M'|>|M|$,
then at least one connected component is in the form of such a path $P$ with 
odd length.
But such a path $P$ is actually an $M-augmenting$ path because,
as $E_{odd}\in M'$ the endpoints of $P$ are not covered by matching $M$.
Thus we can conclude that if $M$ is not a maximum matching, 
there will be an $M$-augmenting path.
So if no $M$-augmenting path exists $M$ must be a maximum matching.

\section{}

\end{document}
