\documentclass[12pt,letterpaper]{article}
\usepackage{fullpage}
\usepackage[top=2cm, bottom=4.5cm, left=2.5cm, right=2.5cm]{geometry}
\usepackage{amsmath,amsthm,amsfonts,amssymb,amscd}
\usepackage{lastpage}
\usepackage{enumerate}
\usepackage{fancyhdr}
\usepackage{mathrsfs}
\usepackage{xcolor}
\usepackage{graphicx}
\usepackage{listings}
\usepackage{hyperref}
\usepackage{mathtools}
\usepackage{xfrac}


\hypersetup{%
  colorlinks=true,
  linkcolor=blue,
  linkbordercolor={0 0 1}
}
\linespread{1.1}
 
\renewcommand\lstlistingname{Algorithm}
\renewcommand\lstlistlistingname{Algorithms}
\def\lstlistingautorefname{Alg.}


\lstdefinestyle{Python}{
    language        = Python,
    frame           = lines, 
    basicstyle      = \footnotesize,
    keywordstyle    = \color{blue},
    stringstyle     = \color{green},
    commentstyle    = \color{red}\ttfamily
}

\setlength{\parindent}{0.0in}
\setlength{\parskip}{0.05in}

% Edit these as appropriate
\newcommand\course{Stochastic Processes}
\newcommand\hwnumber{2}                  % <-- homework number
\newcommand\NetIDa{SUN Yilin}           % <-- NetID of person #1
\newcommand\NetIDb{520030910361}           % <-- NetID of person #2 (Comment this line out for problem sets)

\pagestyle{fancyplain}
\headheight 35pt
\lhead{\NetIDa}
\lhead{\NetIDa\\\NetIDb}                 % <-- Comment this line out for problem sets (make sure you are person #1)
\chead{\textbf{\Large Homework \hwnumber}}
\rhead{\course \\ \today}
\lfoot{}
\cfoot{}
\rfoot{\small\thepage}
\headsep 1.5em

\begin{document}

\section{Optimal Coupling}
\subsection{Basic Ideas}
To reach the lower bound $D_{TV}(\mu,\nu)$, the intuition is that we maximize the terms where $X=Y$. Denote our optimal coupling by $\omega^*$, then $\forall (x,y)\in\Omega^2$ such that $x=y$, the maximum of $\omega^*(x,y)$ can only be $\min\{\mu(x),\nu(y)\}$. As we want to maximize these cases, we can directly set it as $\min\{\mu(x),\nu(y)\}$.\\
Then we need a formula to define $\omega^*(x,y)$ where $x\neq y$. Here the intuition is that $\omega^*(x,y)$ should be some constant times $\max\{\mu(x)-\nu(x),0\}\cdot\max\{\nu(y)-\mu(y),0\}$. So up to here, I construct a coupling as follows:\\
$$\omega^*(x,y)=\begin{cases}
\min\{\mu(x),\nu(y)\},\quad x=y\\
C\cdot\max\{\mu(x)-\nu(x),0\}\cdot\max\{\nu(y)-\mu(y),0\},\quad o.w.
\end{cases}$$
Now I will determine $C$ so that $\omega^*$ is indeed a valid coupling, and then show $Pr_{(X,Y)\sim\omega^*}(X\neq Y)$ is actually $D_{TV}(\mu,\nu)$.
\subsection{Determine $C$ and proof of valid coupling}
Define $A=\{x\in\Omega|\mu(x)\geq\nu(x)\}$. It follows that $\bar{A}=\{x\in\Omega|\mu(x)<\nu(x)\}$. We now calculate the marginal distribution of $X$ under $\omega^*$.\\
\begin{align}
  \forall x\in A,\quad \sum_{y\in\Omega}\omega^*(x,y)&=\sum_{y\in\Omega\wedge y=x}\omega^*(x,y)+\sum_{y\in\Omega\wedge y\neq x}\omega^*(x,y)\\
  &=\min\{\mu(x),\nu(x)\}+\sum_{y\in\Omega\wedge y\neq x}\omega^*(x,y)\\
  &=\nu(x)+C\max\{\mu(x)-\nu(y),0\}\sum_{y\in\Omega\wedge y\neq x}\max\{\nu(y)-\mu(y),0\}\\
  &=\nu(x)+C(\mu(x)-\nu(x))\sum_{y\in\bar{A}}(\nu(y)-\mu(y))
\end{align}
Note that $\sum_{y\in\bar{A}}(\nu(y)-\mu(y))=D_{TV}(\mu,\nu)$ by our definition of $\bar{A}$, so by setting $C=1/D_{TV}(\mu,\nu)$ we can get 
\begin{align}
  \forall x\in A,\quad \sum_{y\in\Omega}\omega^*(x,y)&=\nu(x)+\mu(x)-\nu(x)\\
  &=\mu(x)
\end{align}
We use the same $C$ and consider the cases where $x\in\bar{A}$. Similarly
\begin{align}
  \forall x\in \bar{A},\quad \sum_{y\in\Omega}\omega^*(x,y)&=\min\{\mu(x),\nu(x)\}+\sum_{y\in\Omega\wedge y\neq x}\omega^*(x,y)\\
  &=\mu(x)+D_{TV}(\mu,\nu)\max\{\mu(x)-\nu(x),0\}\sum_{y}\max\{\nu(y)-\mu(y),0\}\\
  &=\mu(x)+D_{TV}(\mu,\nu)\cdot 0\cdot\sum_{y}\max\{\nu(y)-\mu(y),0\}\\
  &=\mu(x)
\end{align}
Now we have proved such a coupling satisfies that the marginal distribution of $X$ is indeed $\mu(x)$. The same argument works for $Y$ as well. So by setting $C=1/D_{TV}(\mu,\nu)$ we actually constructed 
$$\omega^*(x,y)=\begin{cases}
  \min\{\mu(x),\nu(y)\},\quad x=y\\
  \frac{1}{D_{TV}(\mu,\nu)}\cdot\max\{\mu(x)-\nu(x),0\}\cdot\max\{\nu(y)-\mu(y),0\},\quad o.w.
  \end{cases}$$
Now we only need to show $Pr_{(X,Y)\sim\omega^*}(X\neq Y)$ is actually $D_{TV}(\mu,\nu)$.
\subsection{Proof of optimal coupling}
We show by calculating $Pr_{(X,Y)\sim\omega^*}(X=Y)$.\\
\begin{align}
  Pr_{(X,Y)\sim\omega^*}(X=Y)&=\sum_{x\in\Omega}\min\{\mu(x),\nu(x)\}\\
  &=\sum_{x\in A}\nu(x)+\sum_{x\in\bar{A}}\mu(x)\\
  &=\nu(A)+\mu(\bar{A})\\
  &=\nu(A)+(1-\mu(A))\\
  &=1-(\mu(A)-\nu(A))\\
  &=1-D_{TV}(\mu,\nu)
\end{align}
So $Pr_{(X,Y)\sim\omega^*}(X\neq Y)=D_{TV}(\mu,\nu)$
\newpage 

\section{Stochastic Dominance}

\section{Total Variation Distance is Non-Increasing}

\section{Acknowledgements}
The inspiration of constructing the optimal coupling is from the lecture notes of a MIT course 6.896 Probability and Computation. Actually my construction is not entirely like its demonstration, but the core ideas are the same.
\end{document}
