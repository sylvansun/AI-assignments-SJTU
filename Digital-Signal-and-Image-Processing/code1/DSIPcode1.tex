\documentclass[12pt,letterpaper]{article}
\usepackage{fullpage}
\usepackage[top=2cm, bottom=4.5cm, left=2.5cm, right=2.5cm]{geometry}
\usepackage{amsmath,amsthm,amsfonts,amssymb,amscd}
\usepackage{lastpage}
\usepackage{enumerate}
\usepackage{fancyhdr}
\usepackage{mathrsfs}
\usepackage{xcolor}
\usepackage{graphicx}
\usepackage{listings}
\usepackage{hyperref}
\usepackage{mathtools}
\usepackage{xfrac}


\hypersetup{%
  colorlinks=true,
  linkcolor=blue,
  linkbordercolor={0 0 1}
}
\linespread{1.1}
 
\renewcommand\lstlistingname{Algorithm}
\renewcommand\lstlistlistingname{Algorithms}
\def\lstlistingautorefname{Alg.}


\lstdefinestyle{Python}{
    language        = Python,
    frame           = lines, 
    basicstyle      = \footnotesize,
    keywordstyle    = \color{blue},
    stringstyle     = \color{green},
    commentstyle    = \color{red}\ttfamily
}

\setlength{\parindent}{0.0in}
\setlength{\parskip}{0.05in}

% Edit these as appropriate
\newcommand\course{DSIP}
\newcommand\hwnumber{1}                  % <-- homework number
\newcommand\NetIDa{SUN Yilin}           % <-- NetID of person #1
\newcommand\NetIDb{520030910361}           % <-- NetID of person #2 (Comment this line out for problem sets)

\pagestyle{fancyplain}
\headheight 35pt
\lhead{\NetIDa}
\lhead{\NetIDa\\\NetIDb}                 % <-- Comment this line out for problem sets (make sure you are person #1)
\chead{\textbf{\Large Programming \hwnumber}}
\rhead{\course \\ \today}
\lfoot{}
\cfoot{}
\rfoot{\small\thepage}
\headsep 1.5em

\begin{document}

\section{}
By generating a PSF matrix and doing matrix-level convolution we can get the matrix of the blurred image. However when we are trying to 
show the blurred image we must notice that the built-in function of MATLAB does not support float not in the interval
$[0,1]$, i.e. it treats any float larger than 1 as pure white pixel. 
\\
Our original matrix is a uint8 matrix with its elements in $[0,255]$, whose data type becomes float after doing
convolution with PSF matrix. If directly output without normalization we can only get a picture which is mostly white.
\section{}
When adding white gaussian noise to the blurred image, we must note that the $awgn$ function requires us to set a parameter which is 
$'measured'$.  When adding this parameter, the function will automatically caluculate the 
power of a given signal and add noise accordingly.\\
After we add noise with different SNR, we can directly see their difference.
\section{}
\subsection{Direct deconvolution}
Firstly we do deconvolution directly without knowing that there is gaussian noise in our picture,
and it turns out that we get nothing as shown in the pictures attached below. This tells us that 
gaussian noise, however slight, actually violates the process of deconvolution. 
\subsection{Wiener's Method}
Then we use Wiener's Method which minimizes the mean-square of estimated error. To do this, we must know the SNR(or NSR if you want) in advance.
And it turns out Wiener's Method works well under different SNRs.
\subsection{Lucy-Richardson's Method}
The links provided by teacher also recommends a Lucy-Richardson's Method. The advantage of this method is that it can do deconvolution without knowing the SNR.
The drawback is that it works worse when the SNR is decreasing. 
\end{document}
