\documentclass[12pt,letterpaper]{article}
\usepackage{fullpage}
\usepackage[top=2cm, bottom=4.5cm, left=2.5cm, right=2.5cm]{geometry}
\usepackage{amsmath,amsthm,amsfonts,amssymb,amscd}
\usepackage{lastpage}
\usepackage{enumerate}
\usepackage{fancyhdr}
\usepackage{mathrsfs}
\usepackage{xcolor}
\usepackage{graphicx}
\usepackage{listings}
\usepackage{hyperref}
\usepackage{mathtools}
\usepackage{xfrac}

\hypersetup{%
  colorlinks=true,
  linkcolor=blue,
  linkbordercolor={0 0 1}
}
\linespread{1.1}
 
\renewcommand\lstlistingname{Algorithm}
\renewcommand\lstlistlistingname{Algorithms}
\def\lstlistingautorefname{Alg.}

\lstdefinestyle{Python}{
    language        = Python,
    frame           = lines, 
    basicstyle      = \footnotesize,
    keywordstyle    = \color{blue},
    stringstyle     = \color{green},
    commentstyle    = \color{red}\ttfamily
}

\setlength{\parindent}{0.0in}
\setlength{\parskip}{0.05in}

% Edit these as appropriate
\newcommand\course{ConvexOptimization}
\newcommand\hwnumber{13}                  % <-- homework number
\newcommand\NetIDa{SUN Yilin}           % <-- NetID of person #1
\newcommand\NetIDb{520030910361}           % <-- NetID of person #2 (Comment this line out for problem sets)

\pagestyle{fancyplain}
\headheight 35pt
\lhead{\NetIDa}
\lhead{\NetIDa\\\NetIDb}                 % <-- Comment this line out for problem sets (make sure you are person #1)
\chead{\textbf{\Large Homework \hwnumber}}
\rhead{\course \\ \today}
\lfoot{}
\cfoot{}
\rfoot{\small\thepage}
\headsep 1.5em

\begin{document}

\section{}
\subsection*{(a).}
It’s easy to see that $f(x)$ is monotonically increasing, thus the optimal solution is $x^*=0$ and the optimal value is $f^*=$log2.
\subsection*{(b).}
The dual function is $$\phi (\mu)=\inf_{x\in\mathbb{R}}[\log(1+e^x)-\mu x]$$
Then we can easily see that when $\mu\notin [0,1]$, $\phi(\mu)$ is unbounded below. When $\mu\in [0,1]$,we can calculate the explicit expression of $\phi(\mu)$. So the dual function is 
$$\phi(\mu)=\begin{cases}(\mu-1)\log(1-\mu)-\mu\log\mu, \mu\in[0,1]\\
-\infty ,\rm otherwise
\end{cases}$$
The dual problem is $$\max_{\mu} \quad\phi(\mu) $$
$$\rm s.t. \quad \mu\geq \rm0$$
\subsection*{(c).}
We only need to consider $\mu\in[0,1]$, where $\phi'(\mu)=\log(\frac{1-\mu}{\mu})$. We can easily see that $\phi'$ is monotonically decreasing and $\phi'(\frac{1}{2})=0$. So the dual optimal solution is $\mu^*=\frac{1}{2}$ and the dual optimal value is $\phi^*=\log2$. The strong duality holds.
\section{}
\subsection*{(a).}
The Lagrange dual function is $$\phi(\mu_1,\mu_2)=\inf_{\boldsymbol{x}\in\mathbb{R}^2}[(1+\mu_1+\mu_2)x_1^2-2(\mu_1+\mu_2)x_1+(1+\mu_1+\mu_2)x_2^2-2(\mu_1-\mu_2)x_2+\mu_1+\mu_2]$$
We can easily see that when $\mu_1+\mu_2+1\leq0$, $\phi$ is unbounded below. When $\mu_1+\mu_2+1>0$, we can calculate the explicit expression of $\phi(\mu_1,\mu_2)$. So the dual function is $$
\phi(\mu_1,\mu_2)=
\begin{cases}1-\frac{(\mu_1-\mu_2)^2+1}{\mu_1+\mu_2+1}, \mu_1+\mu_2+1>0\\
-\infty, \rm otherwise
\end{cases}
$$
The dual problem is 
$$\max_{\mu_1,\mu_2}\quad \phi(\mu_1,\mu_2)\\$$
$$\rm s.t. \quad  \mu_1,\mu_2 \geq \rm 0$$
\subsection*{(b).}
We only need to consider the case where $\mu_1+\mu_2+1>0$. Its easy to see $\phi<1$ when $\mu_1+\mu_2+1>0$. Also note that $\phi$ goes to 1 as $\mu_1=\mu_2$ and go to positive infinity together. So the dual optimal value is $\phi^*=1$. \\
And we already know that the primal optimal value is $f^*=1$, so strong duality holds.
\subsection*{(c).}
The Slater's condition does not hold as there is no point in int$D$ which is strictly feasible. Yet the strong duality still holds, which means that Slater's condition is not necessary for strong duality.
\subsection*{(d).}
It is not attained by any dual feasible point because as we have shown in part (b), its optimal value is reached only when $\mu_1$ and $\mu_2$ both go to positive infinity. This is expected because in Problem 2(b) we have shown that the optimal point $\boldsymbol{x}^*$ is not a regular point.
\section{}
\subsection*{(a).}
Follow the expression given by this problem, we can derive the explicit expression of $\phi(\mu)$, which is 
$$\phi(\mu)=\begin{cases}
\mu, \quad \mu\leq0\\
\mu-\frac{4}{3\sqrt{3}}\mu^{\frac{3}{2}},\quad \mu >0
\end{cases}
$$
\subsection*{(b).}
When $\mu>0$, $\phi'(\mu)=1-\frac{2}{\sqrt{3}}\mu^{\frac{1}{2}}$. So we can easily see that $\phi(\mu)$ reaches its maximum at point $\mu^*=\frac{3}{4}$. The dual optimal value is $\phi^*=\frac{1}{4}$.
\subsection*{(c).}
Firstly we can see that $f(\frac{1}{2},\frac{1}{2})=\frac{1}{4}$ where $\boldsymbol{x}=(\frac{1}{2},\frac{1}{2})$ is feasible. Also note that weak duality always holds, meaning that $f^*\geq\phi^*$. Thus $f^*\geq\frac{1}{4}$. Then we can say that $f^*$ is actually $\frac{1}{4}$ as it is reached by point $\boldsymbol{x}=(\frac{1}{2},\frac{1}{2})$.
\subsection*{(d).}
In this case we can easily see that the dual function $$\phi(\boldsymbol{\mu})=\inf_{\boldsymbol{x}\in\mathbb{R}}[x_1^3+x_2^3+\mu_1(1-x_1-x_2)-\mu_2x_1-\mu_3x_2]
$$
is unbounded from below. Thus the dual function is actually $$
\phi(\boldsymbol{\mu})=-\infty$$
The strong duality does not hold.
\section{}
\subsection*{(a).}
Here we have $f(\boldsymbol{w}^*,b^*)=f^*$ and $\phi(\boldsymbol{\mu}^*)=\phi^*$. Also, note that here all inequality constraints are affine. So here feasibility means Slater's condition holds for the primal problem. Then by Slater's Theorem strong duality holds, i.e. $f^*=\phi^*$. Under these conditions, KKT conditions hold.\\
So by complementary slackness for any $i$ with $\mu_i^*>0$, $g_i(\boldsymbol{w}^*,b^*)$ must be zero. Which means $y_i(\boldsymbol{x}_i^T\boldsymbol{w}^*+b^*)=1$. Note that $y_i^2=1$ for any $i$, so $b^*=y_i-\boldsymbol{x}_i^T\boldsymbol{w}^*$. 
\subsection*{(b).}
The output and the figure are given below.\\
primal optimal:\\
 w = [-1.09090908  1.45454545]\\
 b = [-0.09090911]\\

dual optimal:\\
 mu = [1.65289255e+00 0.00000000e+00 0.00000000e+00 0.00000000e+00
 0.00000000e+00 0.00000000e+00 0.00000000e+00 1.65289254e+00
 0.00000000e+00 0.00000000e+00 7.11813924e-09 0.00000000e+00
 0.00000000e+00]\\

\begin{figure}[h]
\centering
\includegraphics[]{svm.pdf}
\caption{SVM result}
\label{trandgap1a}
\end{figure}
\end{document}
